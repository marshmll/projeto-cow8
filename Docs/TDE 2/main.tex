\documentclass[11pt]{article}

\usepackage[portuguese]{babel}
\usepackage[a4paper,top=2cm,bottom=2cm,left=3cm,right=3cm,marginparwidth=1.75cm]{geometry}
\usepackage{amsmath}
\usepackage{graphicx}
\usepackage{float}
\usepackage[colorlinks=true, allcolors=blue]{hyperref}
\usepackage{listings}
\usepackage{hyphenat}
\usepackage{xcolor}
\usepackage{booktabs}
\usepackage{caption}
\usepackage[utf8]{inputenc}
\usepackage{array}

\title{\textbf{Escolha e Implementação de um Framework Front-end para o MVP Cow8}}
\author{
    Renan da Silva Oliveira Andrade (\texttt{renan.silva3@pucpr.edu.br})\\
    Ricardo Lucas Kucek (\texttt{ricardo.kucek@pucpr.edu.br})\\
    Pedro Senes Velloso Ribeiro (\texttt{pedro.senes@pucpr.edu.br})\\
    Riscala Miguel Fadel Neto (\texttt{riscala.neto@pucpr.edu.br})\\
    Victor Valerio Fadel (\texttt{victor.fadel@pucpr.edu.br})
}

\begin{document}
\maketitle

\section{Introdução}

Contextualização do projeto, objetivos do front-end e importância da escolha de um framework.

\section{Principais Características de Frameworks Front-end}

Discussão sobre modelos, bibliotecas, classes e métodos em frameworks como Bootstrap, Tailwind CSS, React, Vue.js, etc.

\section{Ferramentas e Tecnologias em Frameworks Front-end}

Comparação entre HTML, CSS e JavaScript puros e o uso de frameworks, destacando vantagens e desvantagens.

\section{Avaliação das Necessidades do Projeto}

Análise dos requisitos do projeto (responsividade, usabilidade, desempenho) e como um framework pode atendê-los.

\section{Escolha do Framework Front-end}

Dentre as alternativas presentes no mercado, o MVP Cow8 utiliza o framework front-end \textbf{Tailwind CSS}, pois melhor se adequou aos requisitos de implementação do projeto.

\subsection{Análise Comparativa entre Frameworks Considerados}

Os frameworks considerados para o desenvolvimento foram: \textbf{React.js, Bootstrap e Tailwind CSS}. Foram realizadas análises comparativas primordialmente imparciais sobre as vantagens e desvantagens de se usar cada framework baseadas em 4 pontos críticos: \textbf{Facilidade de uso, Customização, Performance e Integração com o Flask.}

\begin{table}[H]
\centering
\caption{Comparação de Facilidade de Uso}
\begin{tabular}{lccc}
\toprule
\textbf{Framework} & \textbf{Documentação} & \textbf{Comunidade} & \textbf{Curva de Aprendizado} \\
\midrule
React & Excelente & Muito grande & Alta \\
Bootstrap & Muito boa & Enorme & Baixa a média \\
Tailwind CSS & Boa & Crescente & Média \\
\bottomrule
\end{tabular}
\end{table}

\begin{table}[H]
\centering
\caption{Comparação de Capacidade de Customização}
\begin{tabular}{lccc}
\toprule
\textbf{Framework} & \textbf{Flexibilidade} & \textbf{Temas Prontos} & \textbf{Override de Estilos} \\
\midrule
React & Alta & Dependente & Requer CSS adicional \\
Bootstrap & Limitada & Muitos & Possível com Sass \\
Tailwind CSS & Máxima & Poucos & Fácil \\
\bottomrule
\end{tabular}
\end{table}

\begin{table}[H]
\centering
\caption{Comparação de Performance}
\begin{tabular}{lccc}
\toprule
\textbf{Framework} & \textbf{Tamanho do Bundle} & \textbf{Renderização} & \textbf{Otimização} \\
\midrule
React & Grande & Client-side & Code-splitting necessário \\
Bootstrap & Moderado & Server-side & PurgeCSS recomendado \\
Tailwind CSS & Leve & Server-side & Automática \\
\bottomrule
\end{tabular}
\end{table}

\begin{table}[H]
\centering
\caption{Comparação de Integração com Flask}
\begin{tabular}{lccc}
\toprule
\textbf{Framework} & \textbf{Configuração} & \textbf{Templates (Jinja2)} & \textbf{Complexidade} \\
\midrule
React & Complexa & Incompatível direta & Alta \\
Bootstrap & Simples & Perfeita & Baixa \\
Tailwind CSS & Moderada & Funciona & Média \\
\bottomrule
\end{tabular}
\end{table}

Através destas análises, conclui-se que cada framework proporciona resultados únicos. Para cada framework, observa-se que é benéfico nos seguintes casos:

\begin{itemize}
\item \textbf{Bootstrap}: Melhor para projetos simples e rápidos com Flask
\item \textbf{Tailwind CSS}: Ideal para designs customizados sem sacrificar performance
\item \textbf{React}: Recomendado apenas para aplicações complexas com frontend separado
\end{itemize}

\subsection{Motivos da Escolha do Framework Tailwind CSS}

O front-end do MVP Cow8 utiliza-se do framework Tailwind CSS. Existem diversos motivos pelos quais o projeto melhor se adapta ao uso de Tailwind, baseados nas análises apresentadas no capítulo anterior:

\begin{itemize}
\item \textbf{Customização}: O objetivo é criar uma interface altamente customizável e reutilizável, e o Tailwind atende à este requisito.
\item \textbf{Integração com Templates Jinja2}: A combinação do uso dos templates com as classes inline do Tailwind permitem que os estilos das páginas sejam facilmente modificadas onde houver necessidade, sem preocupação com o uso de arquivos de estilos externos que necessitam ser importados de arquivos estáticos.
\item \textbf{Ausência de \textit{Bloatware}}: O Tailwind possui um bundle leve, o que diminui o tempo de renderização e de carregamento das páginas, proporcionando uma experiência de usuário mais avantajada e uma economia de processamento que pode ser direcionada à outras tarefas mais críticas e relevantes, como o processamento de requests de banco de dados e processamento de mensagens WebSocket do broker MQTT.
\item \textbf{Fácil configuração para produção e desenvolvimento via CDN}: A utilização do Tailwind em desenvolvimento é simples, necessitando apenas de um \texttt{<script>} CDN no corpo dos arquivos HTML. Para produção, a configuração também é simples.
\end{itemize}

\section{Implementação do Design Front-end}

\subsection{Estrutura de Diretórios}

O projeto segue uma estrutura de diretórios simples, contando com os Blueprints Flask, módulos de banco de dados e MQTT próprios e \textit{Shell Scripts} utilizados para automatizar o procedimento de inicialização limpa (pois o projeto está sendo desenvolvido em um container Docker, o qual pode ocasionar corrupções).

O arquivo \texttt{requirements.txt} lista todas as dependências de bibliotecas Python do projeto, sendo facilmente instaladas recursivamente no desenvolvimento com o uso de um \textit{Virtual Environment} (venv).

Além disso, os diretórios \texttt{static/} e \texttt{templates/} são os diretórios que contêm os arquivos estáticos e templates HTML, respectivamente.

\begin{verbatim}
    .
    |- api.py
    |- auth.py
    |- clean.sh
    |- database/
    |- __init__.py
    |- main.py
    |- mqtt.py
    |- requirements.txt
    |- static/
    |- templates/

\end{verbatim}

\subsection{Template base e configuração do Tailwind}

\section{Demonstração do Design}

Imagens ou links para demonstração do front-end desenvolvido.

\section{Conclusão}

Reflexão sobre as vantagens (produtividade, consistência visual) e desafios (curva de aprendizado, tamanho do bundle) do framework escolhido.

% \bibliographystyle{apalike}
% \bibliography{sample}

\end{document}
