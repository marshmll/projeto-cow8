\documentclass[11pt]{article}

\usepackage[portuguese]{babel}
\usepackage[a4paper,top=2cm,bottom=2cm,left=3cm,right=3cm,marginparwidth=1.75cm]{geometry}
\usepackage{amsmath}
\usepackage{graphicx}
\usepackage{float}
\usepackage[colorlinks=true, allcolors=blue]{hyperref}
\usepackage{listings}
\usepackage{hyphenat}
\usepackage{xcolor}

\title{\textbf{Escolha e Implementação de um Framework Front-end para o MVP Cow8}}
\author{
    Renan da Silva Oliveira Andrade (\texttt{renan.silva3@pucpr.edu.br})\\
    Ricardo Lucas Kucek (\texttt{ricardo.kucek@pucpr.edu.br})\\
    Pedro Senes Velloso Ribeiro (\texttt{pedro.senes@pucpr.edu.br})\\ 
    Riscala Miguel Fadel Neto (\texttt{riscala.neto@pucpr.edu.br})\\
    Victor Valerio Fadel (\texttt{victor.fadel@pucpr.edu.br})
}

\begin{document}
\maketitle

\section{Introdução}

Contextualização do projeto, objetivos do front-end e importância da escolha de um framework.

\section{Principais Características de Frameworks Front-end}

Discussão sobre modelos, bibliotecas, classes e métodos em frameworks como Bootstrap, Tailwind CSS, React, Vue.js, etc.

\section{Ferramentas e Tecnologias em Frameworks Front-end}

Comparação entre HTML, CSS e JavaScript puros e o uso de frameworks, destacando vantagens e desvantagens.

\section{Avaliação das Necessidades do Projeto}

Análise dos requisitos do projeto (responsividade, usabilidade, desempenho) e como um framework pode atendê-los.

\section{Escolha do Framework Front-end}

Justificativa da escolha (escolhemos o Tailwind CSS por sua flexibilidade, etc.).

\section{Implementação do Design Front-end}

Detalhes da implementação, estrutura de pastas, componentes utilizados e integração com o back-end (Flask).

\section{Demonstração do Design}

Imagens ou links para demonstração do front-end desenvolvido.

\section{Conclusão}

Reflexão sobre as vantagens (produtividade, consistência visual) e desafios (curva de aprendizado, tamanho do bundle) do framework escolhido.

% \bibliographystyle{apalike}
% \bibliography{sample}

\end{document}
